\section{Context}

Jerusalem, the ancient and historically significant city, has been a treasure trove of architectural wonders from different periods in history. \\
Among these, the Ayyubid and Mamluk eras stand out as a testament to the rich cultural and artistic heritage of the region. These periods, spanning from the 12th to the 16th centuries, witnessed the construction of numerous architectural masterpieces that shaped the city's skyline and influenced architectural styles throughout the Islamic world. \\
Preserving and studying these architectural gems is of paramount importance to our understanding of Jerusalem's historical development and its cultural significance.

\section{Motivation}
The study and documentation of architectural heritage have traditionally relied on manual methods, such as extensive field surveys, handwritten notes, and photographs. \\
While these approaches have provided valuable insights, they often suffer from limitations in terms of scalability, accessibility, and ease of analysis. In the case of the Ayyubid and Mamluk architectures of Jerusalem, there is a wealth of information locked away in books and documents that are not easily searchable or accessible for in-depth analysis. \\
Therefore, there is a pressing need to leverage modern technologies, such as computer vision, Machine Learning, Natural Language Processing, and database management to transform this valuable information into a digital format and make it readily available for researchers, scholars, and the general public.


\section{Goals}
The primary goal of this project is to create a comprehensive and interactive digital catalogue of Jerusalem's Ayyubid and Mamluk architectures. By leveraging computer vision techniques, we aim to extract architectural plans, drawings, photographs, and textual descriptions from the book \textit{Mamluk Jerusalem: An Architectural Study.} \parencite{burgoyne1987mamluk}\\


Through optical character recognition (OCR), layout recognition, and NLP algorithms, we will convert these images and texts into structured and searchable data. Furthermore, by utilizing machine learning algorithms, we will enhance the accuracy and efficiency of the data extraction process.

Additionally, we will develop a robust and scalable database management system to store, organize, and manage the extracted data. This will facilitate easy access, efficient retrieval, and cross-referencing of architectural elements, styles, and historical contexts. Finally, we will leverage data visualization techniques to create an intuitive and interactive interface that allows users to explore the digital catalogue, visualize architectural features, and delve into the historical narratives embedded within Jerusalem's Mamluk architectures.

By achieving these goals, this project aims to bridge the gap between traditional architectural studies and modern technological advancements. The resulting digital catalogue will provide an invaluable resource for researchers, historians, architects, and anyone interested in the architectural heritage of Jerusalem. It will enable new avenues of research, facilitate comparative analysis, and contribute to the preservation and understanding of Jerusalem's rich cultural history.

