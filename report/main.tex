\documentclass[11pt,titlepage]{report}
\usepackage{Preamble}
\begin{document}


\begin{titlepage}
    \newgeometry{margin=3cm}
	\centering
    \includegraphics[width=0.5\linewidth]{epfl.png}\\[0.25cm] 	% University Logo
    \textsc{\LARGE École Polytechnique Fédérale de Lausanne}\\ \vspace{\fill}
    \textbf{\textsc{\fontsize{25}{25}\selectfont A digital catalogue of Jerusalem’s Ayyubid and Mamluk architectures}}\\ \vspace{\fill}		
	\textsc{\LARGE EPFL DHLAB}\\[0.4cm]
	\rule{\linewidth}{0.2 mm} \\[0.5 cm]
	\textbf{Author:} Youssef Mamlouk \\ [0.5 cm]
 \textbf{Supervisor:} Beatrice Vaienti \\
\textbf{Professor:} Pr. Frederic Kaplan \\[1.5cm] \today
\end{titlepage}
\restoregeometry

\thispagestyle{numberonly}
\begin{summary}
\section*{Abstract}

This report presents a comprehensive approach to creating a digital catalog of Jerusalem's Ayyubid and Mamluk architectures. \\

The project utilizes computer vision, machine learning, natural language processing, and database management techniques to convert the book \textit{Mamluk Jerusalem: An Architectural Study} into a digital database. The database is then visualized through an interactive leaflet map, enabling users to explore the buildings and their historical information. \\

The project showcases the successful extraction of architectural elements, accurate OCR results, and efficient database management. \\
The digital catalog serves as a valuable resource for researchers, enthusiasts, and preservationists interested in studying and appreciating the architectural heritage of Jerusalem.
\end{summary}

\clearpage

\tableofcontents


\chapter{Introduction}
\subfile{1-Introduction}

\chapter{Methods}
\subfile{2-Methods}

\chapter{Results}
\subfile{3-Results}

\chapter{Discussion}
\subfile{4-Discussion}

\chapter{Conclusion}
\subfile{5-Conclusion}

\clearpage
\pagestyle{numberonly}

\printbibliography

\end{document}
